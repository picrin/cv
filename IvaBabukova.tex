\documentclass{tccv}
\usepackage[english]{babel}
\usepackage{mathtools}
\usepackage{relsize}
\begin{document}
\hyphenation{pre-par-ing}
\setlength{\emergencystretch}{3em}
\part{Iva Babukova}

\section{Projects}

\begin{eventlist}

\item{Encrypt}
     {Password security awareness game in your browser}
     {Ever increasing computational power allows the humankind to tackle ever more challenging tasks. . This simple and engaging game is trying to address the problem of lack of awareness on password security amongst users. \par\medskip You can play it here: \href{encryptgame.co.uk}{encryptgame.co.uk}.
     }

\item{Strongly Connected Divide and Conquer}
   {Big data meets parallel graph algorithms}
   {Graphs have always fascinated me, and have been my passion ever since I learned about them in my first year algorithms course. In this project, which I contributed to in my free time, I have contributed to one of the first open source implementations of Strongly Connected Divide and Conquer -- a graph algorithm frequently used as a first step in analysis of large graphs. \par\medskip
   Source at
   \href{http://github.com/picrin/SCDC}{github.com/picrin/SCDC}
   }

\item{Subgraph Query Processing methods}
    {Design, test and implementation in GraphX}
    {GraphX is a distributed in-memory graph processing system. As part of my degree I have chosen to design, implement, and test a set of algorithms for subgraph queries in GraphX, under the supervision of \href{http://www.gla.ac.uk/schools/computing/staff/petertriantafillou}{professor Peter Triantafillou}}

\end{eventlist}

\section{Education}

\begin{yearlist}

\item[University of Glasgow]
     {2012 -- 2017}
     {Master in Science}
     {Glasgow, UK}

\item[Sofia Mathematics HS]
     {2007 -- 2012}
     {Extended mathematics}
     {Sofia, Bulgaria}

\item[107 Primary School]
     {2000 -- 2007}
     {Extended mathematics}
     {Sofia, Bulgaria}
\end{yearlist}

\newpage

\personal
    [www.github.com/ivababukova]
    {276 Paisley Road West\newline G51 1BJ Glasgow}
    {+44 7519 522 305}
    {ibabukova@gmail.com}



\section{Experience}

\begin{skillist}

\item{Organiser} {In May 2013 I co-founded GUTS -- \href{http://gutechsoc.com}{Glasgow University Tech Society}. I have served as the treasurer of the society for since its foundation. I am responsible for budgeting of the society's expenses, carrying out numerous purchases for the society and helping to organise events. My biggest achievement within society were co-organisation of three annual \href{http://storify.com/Eventhread/gu-hackaton}{hackathons} in years 2013-2015. The first hackathon attracted about 50 participants and a few corporate sponsors. The last one was a truly global event, with over 250 participants in Scotland, and about 100 in Glasgow University's off-shore Campus in Singapore, over a dozen of corporate sponsors, and a budget of about 10 000 pounds. The Society has been a great hands-on opportunity to learn and practice teamwork in various, often stressful situations.}
\item{Intern at SIE} {Through my activity with the Society I built up a broad network of connections with Glasgow's tech industry, and I was offered a part-time job with Scottish Institute for enterprise. As part of this employment, I organise weekly meetings with members of the Scottish start-up scene, and offer advice to students with start-up ideas. }

\item{Summer Technology Intern at JP Morgan} {My internship at JP Morgan was my first encounter with software engineering at large, and from beginning it was difficult. During my first day I found out that I was assigned a role of "business analyst", which meant no coding. Luckily, and because my line manager turned out to be a wonderful person, my technical excellence was quickly discovered, and I was reassigned to a team working on a Java back-end/C\# front-end options trading application, and I discovered in myself a passion for real-life software engineering.}
\newpage
\item{Teaching Python at University of Glasgow} {Since the 3rd year of my study I have been employed by University of Glasgow to tutor 1st year students programming in Python. I was responsible for helping out students with their programming tasks during their lab sessions and answering their questions about the language and the algorithms underlying their assignments.}

\item{Teaching maths in primary school} {In my final year of high school I volunteered to teach students extracurricular material in mathematics and combinatorics. My responsibilities included preparing lessons and delivering them to students, giving them homework and checking their individual progress. Although the course was not compulsory, and the students could stop attending at any time, I was surprised to discover that at the end of the year I had more students attending my class than at the start.}

\end{skillist}

\newpage


\section{Skills}

\begin{skillist}
\item{Team player} {My major strength in the team is being empathic -- I sense people's feelings and emotions and I know how to influence their decision making without stepping on their egos. Carefully manoeuvring I managed to convince my team to choose HTML5 for the 3rd year team project despite other technologies being preferred by various team members. This turned out to be a great decision, since the game can be now easily accessed and played by anybody in the browser. As I prefer doing over talking, I significantly contributed to the game's source base a record of which is publicly accessible \href{https://github.com/ivababukova/TP3-Security-game}{on github}.}

\item{Adaptable} {My ability to adapt fast to changing environments has always helped me in various circumstances. The time when I needed it most was when I moved away from home to study at the University of Glasgow three years ago. At Glasgow I was exposed to completely new environment and I was far away from my friends and family. My ability to cope effectively with change really helped me to adapt fast and achieve excellent results at the university, to find new friends and co-found the Glasgow University Tech Society.}

\item{Hard worker} {I would not be able to support myself through part-time jobs and achieve great results at University if not my ability to work hard, work efficiently, and stay focused on delivering the best results regardless of the task.}

\end{skillist}
\newpage
\section{Tools}

\begin{skillist}

\item{Python} {I have used python since the 1st year of my degree. I accomplished several pieces of university coursework in Python and I also developed a small game similar to the classic \href{http://en.wikipedia.org/wiki/Breakout_(video_game)}{Breakout arcade game}.}

\item{JavaScript, CSS, HTML} {I learned HTML5 outside of university, and used the technology mostly in hackathons and in my personal projects. In the 3rd year, I lobbied for HTML5 to be the technology used for my team project, and I managed to convince the team to implement \textit{Encrypt} in a javascript game engine \textit{Phaser.io}.}

\item{Java}
{I wrote Java code mainly for coursework and projects at university. I have more than two years experience with object-oriented programming in Java. I know multiple design patterns, and can design UML diagrams.}
 

\item{C\#}
{C\# was a novelty which I had to quickly learn for my work at JP Morgan. Although I haven't used it since then, I hope I'll have an opportunity to use it again, because it appealed to me as better Java.}

\item {Languages } {native Bulgarian, fluent English, intermediate German and basic Polish}
\end{skillist}

\end{document}

