\documentclass{tccv}
\usepackage[english]{babel}
\usepackage{mathtools}
\usepackage{relsize}
\begin{document}
\setlength{\emergencystretch}{3em}
\part{Iva Babukova}

\section{\LARGE Experience \normalsize}

\begin{skillist}

\item{Programming Tutor at University of Glasgow} {In the 3rd year of my undergraduate degree I worked as a tutor at University of Glasgow teaching 1st year students programming in Python. I was responsible for helping students complete their programming task during the lab session as well as answering any questions about the language they had.}

\item{Teaching maths at primary school} {During my final year at high school I volunteered to teach students extra curriculum material in mathematics and combinatorics. My responsibilities included preparing each lesson and delivering it to the students. I also had to give them homeworks and check their individual progress. Although the course was not compulsory, and the students knew they could stop attending at any time, I was surprised to discover that at the end of the year I had more students attending my class than at the start.}

\item{Part time jobs} {I worked part time in various restaurants and companies in order to support myself during my studies. I worked as a bartender at \href{http://www.akbars.co.uk/glasgow}{Akbar's}, as a waitress at \href{http://www.tripadvisor.co.uk/Restaurant_Review-g186534-d4087172-Reviews-Riccardo_s_Italian_Kitchen-Glasgow_Scotland.html}{Riccardo's} italian restaurant and as a cleaner at \href{http://www.csmfacilities.co.uk/cleaning-company-glasgow/}{CSM}. Because of these jobs I was not only able to fully sustain myself, but I gained a lot of experience with working with people and dealing effectively with various types of clients.}

\end{skillist}

\personal
    [www.github.com/ibabukova]
    {220 Wallace Street \newline G5 8AL Glasgow}
    {+44 7519 522 305}
    {ibabukova@gmail.com}

\section{\LARGE Education \normalsize}

\begin{yearlist}

\item[University of Glasgow]
     {2012 -- 2016}
     {Computer Science}
     {Glasgow, UK}

\item[Sofia Mathematics HS]
     {2007 -- 2012}
     {Extended mathematics}
     {Sofia, Bulgaria}

\item[107 Primary School]
     {2007 -- 2010}
     {Extended mathematics}
     {Sofia, Bulgaria}
\end{yearlist}

\section{\LARGE Projects \normalsize}
\hyphenation{pre-par-ing}

\begin{eventlist}

\item{Encrypt}
     {Password security awarness game in your browser}
     {Increasing computer performance allows the humankind to tackle ever more challenging tasks. At the same time password cracking has never been so easy. This simple and engaging game is trying to address the problem of lack of awarness on password security amongst users. \par\medskip You can play it here: \href{encryptgame.co.uk}{encryptgame.co.uk}.
     }

\item{Strongly Connected Divide and Conquer}
   {Big data meets parallel graph algorithms}
   {Analysis of big and heavily linked datasets is critical in public security. As soon as the investigation begins, every minute matters. That is why it is very important to harness the fastest and most accurate algorithms known to humankind to quickly retrieve relevant information from large graphs. Often the only way to achieve this is by harnessing computer clusters to do the job. This requires using classes of algorithms called parallel algorithms. Both discovery and implementation of these algorithms is very challenging. I'm glad to have contributed to one of the first open source implementations of Strongly Connected Divide and Conquer -- a graph algorithm frequently used as a first step in analysis of large graphs. \par\medskip
   Source at 
   \href{http://github.com/picrin/SCDC}{github.com/picrin/SCDC}
   }

\end{eventlist}

\section{\LARGE Skills \normalsize}

\begin{skillist}
\item{Team player} {My major strength in the team is being empathic -- I sense people's feelings and emotions and I know how to influence their decision making without stepping on their egos. Carefully maneouvering I managed to convince my team to choose HTML5 for the 3rd year team project despite other technologies being preferred by various team members. This turned out to be a great decision, since the game can be now easily accessed and played by anybody in the browser. As I prefer doing over talking, I significantly contributed to the game's source base a record of which is publically accessible \href{https://github.com/ivababukova/TP3-Security-game}{on github}.}

\item{Organiser} {In May 2013 I co-founded GUTS -- \href{http://gutechsoc.com}{Glasgow University Tech Society}. I have served as the treasurer of the society for since its foundation. I am responsible for budgeting of the society's expences, carry out numerous purchases for the society and help to organise events. My biggest achievements within society were co-organisation of two \href{http://storify.com/Eventhread/gu-hackaton}{hackathons}, one in October 2013 and one in October 2014. The first hackathon attracted about 50 participants and over a dozen corporate sponsors. The second hackathon was even more successful, as we had over 120 participants, the students really liked the challenges and we received a lot of sponsorship from various companies.}

\item{Adaptable} {My ability to adapt fast to changing environments has always helped me in various circumstances. The time when I needed it most was when I moved away from home to study at the University of Glasgow three years ago. At Glasgow I was exposed to completely new environment and I was far away from my friends and family. My ability to cope effectively with change really helped me to adapt fast and achieve excellent results at the university, to find new friends and co-found the Glasgow University Tech Society.}

\item{Hard worker} {I would not be able to support myself through part-time jobs and achieve great results at University if not my ability to work hard, work efficiently, and stay focused on delivering the best results regardless of the task.}

\end{skillist}
\newpage
\section{\LARGE Tools \normalsize}

\begin{skillist}

\item{Python} {I have used python since the 1st year of my degree. I accomplished several pieces of university courswork in Python and I also developed a small game similar to the classic \href{http://en.wikipedia.org/wiki/Breakout_(video_game)}{Breakout arcade game}.}

\item{JavaScript, CSS, HTML} {I learned HTML5 outside of university, and used the technology mostly in hackathons and in my personal projects. In the 3rd year, I lobbied for HTML5 to be the technology used for my team project, and I managed to convince the team to implement \textit{Encrypt} in a javascript game engine \textit{Phaser.io}.}

\item{Java}
{I wrote Java code mainly for coursework and projects at university. I have more than two years experience with object-oriented programming in Java. I know multiple design patterns, and can design UML diagrams.}

\item{Standard Query Language (SQL)} % how can I mentioned that I have taught myselt on this? Does it really matter whether I have taught myself on this or not?
 {I have experience with designing, creating, running and developing a relational database application and its associated application software suite. I have used SQL for my own projects as well as for coursework mainly for manipulating and extracting data from a database for data analysis purposes.}
\item {Languages } {fluent English, intermediate German and native Bulgarian}
\end{skillist}

\end{document}
