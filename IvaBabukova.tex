\documentclass{tccv}
\usepackage[english]{babel}
\usepackage{mathtools}
\usepackage{relsize}
\begin{document}
\setlength{\emergencystretch}{3em}
\part{Iva Babukova}

\section{\LARGE Projects \normalsize}
\hyphenation{pre-par-ing}

\begin{eventlist}

\item{Encrypt}
     {Password security awarness game in your browser}
     {Increasing computer performance allows the humankind to tackle ever more challenging tasks. At the same time password cracking has never been so easy. This simple and engaging game is trying to address the problem of lack of awarness on password security amongst users. \par\medskip You can play it here: \href{encryptgame.co.uk}{encryptgame.co.uk}.
     }

\item{Strongly Connected Divide and Conquer}
   {Big data meets parallel graph algorithms}
   {Analysis of big and heavily linked datasets is critical in public security. As soon as the investigation begins, every minute matters. That is why it is very important to harness the fastest and most accurate algorithms known to humankind to quickly retrieve relevant information from large graphs. Often the only way to achieve this is by harnessing computer clusters to do the job. This requires using classes of algorithms called parallel algorithms. Both discovery and implementation of these algorithms is very challenging. I'm glad to have contributed to one of the first open source implementations of Strongly Connected Divide and Conquer -- a graph algorithm frequently used as a first step in analysis of large graphs. \par\medskip
   Source at 
   \href{http://github.com/picrin/SCDC}{github.com/picrin/SCDC}
   }

\end{eventlist}

\personal
    [www.github.com/ibabukova]
    {220 Wallace Street \newline G5 8AL Glasgow}
    {+44 7519 522 305}
    {ibabukova@gmail.com}

\section{\LARGE Education \normalsize}

\begin{yearlist}

\item[University of Glasgow]
     {2012 -- 2016}
     {Computer Science, Mathematics}
     {Glasgow, UK}

\item[Sofia Mathematics HS]
     {2007 -- 2012}
     {Extended maths curriculum}
     {Sofia, Bulgaria}

\item[107 Primary School]
     {2007 -- 2010}
     {Extended maths curriculum}
     {Sofia, Bulgaria}
\end{yearlist}

\section{\LARGE Experience \normalsize}

\begin{skillist}

\item{Programming Tutor at University of Glasgow} {In the 3rd year of my undergraduate degree I worked as a tutor at University of Glasgow teaching 1st year students programming in Python. I was responsible for helping students complete their programming task during the lab session as well as answering any questions about the language they had.}

\item{Teaching maths at primary school} {During my final year at high school I volunteered to teach students extra curriculum material in mathematics and combinatorics. My responsibilities included preparing each lesson and delivering it to the students. I also had to give them homeworks and check their individual progress. Although the course was not compulsory, and the students knew they could stop attending at any time, I was surprised to discover that at the end of the year I had more students attending my class than at the start.}

\item{Part time jobs} {I have worked at various reastaurants as waitress, At Riccardo's Italian Kitchen ... i am able to support myself bla bla bla}

\end{skillist}

\section{\LARGE Skills \normalsize}

\begin{skillist}
\item{Team player} {I work in team very often both for my coursework and for non-academic projects. Programming in a group is often the quickest way for me to catch up with new technology if my team mates are more experienced. As part of the team I always contribute my best skills and knowledge to help the team progress fast. I am specifically good at} % write more where am I within the team? what am I useful in? 

%\item{Team player} {I am a firm believer in pair programming. Both my coursework and non-academic projects regurarly use this technique. In particular, when my partner is more experienced in a specific technology, I find pair programming is the quickest way I can catch up with them. As a part of the team I always try to serve the best of my knowledge, which usually means taking the responsibility for selecting correct algorithms and data structures, reasoning about the correctness of programs, writing critical sections of parallel programs and implementing secure parsers. The above area of expertise stems from my interest in mathematics, which is a part of my degree.}

\item{Organiser} {In May 2013 I co-founded GUTS -- \href{http://gutechsoc.com}{Glasgow University Tech Society}. Currently I serve as a treasurer in the society for a second year. I am responsible for the society budget and I also help with the organisation of various events. My biggest achievements within society were co-organisation of two \href{http://storify.com/Eventhread/gu-hackaton}{hackathons}, one in October 2013 and one in October 2014. The first hackathon attracted about 50 participants and over a dozen corporate sponsors. The second hackathon was even more successful, as we had over 120 participants, the students really liked the challenges and we received a lot of sponsorship from various companies.}

\item{Adaptable} {}

\end{skillist}
\newpage
\section{\LARGE Tools \normalsize}

\begin{skillist}

\item{Python programming language} {I studied this language at university during my first year. I have programmed in Python some of my university coursework and I have also developed a small game very similar to the classic \href{http://en.wikipedia.org/wiki/Breakout_(video_game)}{breakout arcade game}.}

\item{JavaScript, CSS, HTML} {I have taught myself in these languages and I have used them mainly for hackathons and side projects. For example, the password security awareness game project is writen using mainly JavaScript, CSS and HTML.}

\item{Standard Query Language (SQL)} % how can I mentioned that I have taught myselt on this? Does it really matter whether I have taught myself on this or not?
 {I have used SQL mainly for coursework and projects withing the university. I have experience with designing, creating, running and developing a relational database application and its associated application software suite. I have used SQL for my projects mainly for manipulating and extracting data from a database for data analysis purposes.}
\item {Languages } {fluent English, intermediate German and native Bulgarian}
\end{skillist}

\end{document}
